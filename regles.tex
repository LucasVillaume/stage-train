\documentclass[12pt]{article}
\usepackage{paper/package/formal-grammar}
\usepackage{package/mathpartir}
\usepackage{package/algorithm}
\usepackage{package/algpseudocode}
\usepackage{amsmath}
\usepackage{mathtools}
\usepackage{graphicx}
\usepackage[dvipsnames]{xcolor}



\title{Modèle 2}
\author{}

\begin{document}

%%%%% Macro %%%%%%%
\newcommand\concat[2]{#1;#2}
\newcommand\train[4]{(#1, #2, #3, #4)} % #1 id, #2 pos, #3 dir, #4 P
\newcommand\trainfull[4]{\Gamma \cup \train{#1}{#2}{#3}{#4}} 
\newcommand\reg[3]{(#1, #2, #3)} % #1 Events, #2 Jetons, #3 File d'attente

\newcommand\maj[2]{#1 \gets #2} % #1 : dictionnaire, #2 : valeur
\newcommand\majtab[3]{#1[#2] \gets #3}
\newcommand\supprdict[2]{#1' \gets suppr(#1, #2)} % #1 : dictionnaire, #2 : clé
\newcommand\bracket[2]{
    \begin{math}
        \biggl\{
        \begin{array}{l}
            #1\\
            #2
        \end{array}
    \end{math}
}



\maketitle


\section{Définition}

\subsection{Représentation du train}
Dans notre modèle, un train est représenté par un quadruplet $(id, pos, dir, P)$ où :
\begin{itemize}
    \item $id$ : identifiant du train
    \item $pos$ : position du train 
    \item $dir$ : direction du train
    \item $P$ : programme du train
\end{itemize}

\subsection{Représentation du régulateur}
Le régulateur est représenté par un octuplet $(E, T, W)$ où :
\begin{itemize}
    \item $E$ : événements du régulateur
    \item $T$ : tableau de jetons (tokens)
    \item $W$ : file d'attente
\end{itemize}

\subsection{Représentation des aiguillages}
$\Sigma$ / $\theta$ TODO

\subsection{Représentation des feux de signalisation}
Les feux de signalisation sont représentés par un dictionnaire $S$ où chaque clé
est un couple $(numBlock, direction)$ et la valeur associée est l'état du feu de signalisation ("R", "V").

\subsection{Représentation des valeurs "méta"}
M = (B,G) où :
\begin{itemize}
    \item $B$ : buffer, liste des événements en attente
    \item $G$ : garde
\end{itemize}

\subsection{Définition des prédicats et fonctions}

\begin{itemize}
    \item $suiv(pos, dir, switch)$ : renvoie la position suivante, $\varepsilon$ si aucun suivant n'existe
    \item $handleTL(id, pos)$ : renvoie la nouvelle valeur des feux de signalisation
\end{itemize}

%%%%% Grammaire %%%%%%

\section{Grammaire}



\begin{grammar}[Grammaire des trains][h][train]
    \firstcasesubtil{T}{\concat{seq}{T}}{}
    \otherform{\varepsilon }{}
    \firstcasesubtil{seq}{StartUntil(D,N)}{}
    \firstcasesubtil{D}{L \gralt R}{}
    \firstcasesubtil{N}{[ sections ]}{}
    \firstcasesubtil{sections}{\concat{section}{sections} \gralt section \gralt \varepsilon}{} 
    \firstcasesubtil{section}{0 \gralt 1 \gralt ... }{} 
\end{grammar}

\vspace{0.5cm}

\begin{grammar}[Grammaire du régulateur][h][regulateur]
    \firstcasesubtil{R}{stmt || R}{}
    \otherform{\varepsilon}{}
    \firstcasesubtil{stmt}{ev(id_{train}, num_{event}) : atom}{}
    \firstcasesubtil{atom}{incr(j)}{}
    \otherform{att(j, value_{j})}{}
    \otherform{turn(id_{switch}, state)}{}
    \otherform{auth}{}
    \otherform{atom \gralt \varepsilon}{}

\end{grammar}

%%%%%% Règles %%%%%%

\newpage
\section{Règles}

\subsection{Système}

\inferrule
    {\alpha \xRightarrow{\text{!e}} \alpha' \\ \beta \xRightarrow{\text{?e}} \beta' }
    {\alpha||\beta \Rightarrow \alpha' || \beta' }
\vspace{0.5cm}

\subsection{Actions sur le train}

\noindent

Start : 
\inferrule
    { dir \neq D \\ M.G.S = "none"}
    {\trainfull{tid}{pid}{dir}{\concat{SU(D,[\concat{n}{N'}])}{P'}}, M \\ \xRightarrow{\text{!ev(tid,pid)}} \\ \trainfull{tid}{pid}{D}{\concat{SU(D,[\concat{n}{N'}])}{P'}}, M}
\vspace{0.5cm}


Stop :
\inferrule
    { P = \varepsilon \\ dir \neq * \\ M.G.S = "none"}
    {\trainfull{tid}{pid}{dir}{P},M \\ \xRightarrow{\text{!ev(tid,pid)}} \\ \trainfull{tid}{pid}{*}{P},M}
\vspace{0.5cm}

\noindent
Pour les deux prochaines règles, on espère que les valeurs de $n$ et de $s$ sont égales, sinon ça annonce des problèmes futurs.
\vspace{0.5cm}

Until :
    \inferrule
    { suiv(pid, D, \Sigma) = s \\ s \neq \varepsilon \\  S[(pid,dir)] = V \\ M.G.S = "none"}
    {\trainfull{tid}{pid}{D}{\concat{SU(D,[\concat{n}{N'}])}{P'}},S,\Sigma,(B,G) \\ \xRightarrow{\text{!ev(tid,s)}} \\ \trainfull{tid}{s}{D}{\concat{SU(D,[N'])}{P'}},S,\Sigma,(B,\maj{G.S}{"update"})}
\vspace{0.5cm}

Until$_{cons}$ :
    \inferrule
        { suiv(pid, D, \Sigma) = s \\ s \neq \varepsilon \\  S[(pid,dir)] = V \\ G.S = "none"}
        {\trainfull{tid}{pid}{D}{\concat{SU(D,[\concat{n}{\varepsilon}])}{P'}},S,\Sigma,(B,G) \\ \Rightarrow \\ \trainfull{tid}{s}{D}{P'},S,\Sigma,(B,\maj{G.S}{"update"})}
\vspace{0.5cm}

\subsection{Actions sur le régulateur : simulation grands pas avec une approche petits pas}

\noindent

Recv :
\inferrule
    { len(B) = x}
    {(B,G) \\ \xRightarrow{\text{?ev(tid,pid)}} \\ (\majtab{B}{x+1}{(tid,pid)},G)}
\vspace{0.5cm}

StarEvent :
\inferrule
    { len(B) \neq 0 \\ G.S = "none" }
    {(B,G) \\ \Rightarrow \\ (B,\maj{G.S}{"event"})}
\vspace{0.5cm}

Incr$_{bf}$ : %incr qui arrive avant un att
\inferrule
    { Head(B) = (tid, pid) \\ T[j] = x \\ W[(j, x+1)] = \varepsilon \\ G.S = "event" }
    {\reg{E[tid] \owns \concat{incr(j)}{P'}}{T}{W},(B,G) \\ \Rightarrow \\ \reg{\majtab{E}{t_{id}}{P'}}{\majtab{T}{j}{x+1}}{W},(B,G)}
\vspace{0.5cm}

Incr$_{af}$ : %incr qui arrive après un att
\inferrule
    { Head(B) = (tid, pid) \\ T[j] = x \\ getId(W, j, x+1) = wid \\ getPos(W,wid) = wpod \\ RequestS(wid,wpos,E) = req \\ G.S = "event"}
    {\reg{E[tid] \owns \concat{incr(j)}{P'}}{T}{W}, (B,G) \\ \Rightarrow \\ \reg{\majtab{E}{t_{id}}{P'}}{\majtab{T}{j}{x+1}}{Remove(W,(wid,wpos,j,x))},(B,\maj{G.R}{req})}
\vspace{0.5cm}

Att$_{bf}$ : %att qui arrive avant un incr
\inferrule
    { Head(B) = (tid, pid) \\ T[j] = x \\ x \neq y \\ G.S = "event" }
    {\reg{E[tid] \owns \concat{att(j,y)}{P'}}{T}{W}, (B,G) \\ \Rightarrow \\ \reg{\majtab{E}{t_{id}}{P'}}{T}{{Append(W,(tid,pid,j,y))}},(B,G)}
\vspace{0.5cm}

Att$_{af}$ : %att qui arrive après un incr
\inferrule
    { Head(B) = (tid, pid) \\ T[j] = x \\ RequestS(tid,pid,E) = req \\ G.S = "event" }
    {\reg{E[tid] \owns \concat{att(j,x)}{P'}}{T}{W},(B,G) \\ \Rightarrow \\ \reg{\majtab{E}{t_{id}}{P'}}{T}{Remove(W,(tid,pid,j,x))},(B,\maj{G.R}{req})}
\vspace{0.5cm}

Turn : 
\inferrule
    { Head(B) = (tid, pid) \\ G.S = "event"}
    {\reg{E[tid] \owns \concat{turn(sid,state)}{P'}}{T}{W},(B,G),\Sigma \\ \Rightarrow \\ \reg{\majtab{E}{t_{id}}{P'}}{T}{W},(B,G),\majtab{\Sigma}{sid}{state}}
\vspace{0.5cm}

Auth : 
\inferrule
    { Head(B) = (tid, pid) \\ RequestS(tid,pid,E) = req \\ G.S = "event"}
    {\reg{E[tid] \owns \concat{auth}{P'}}{T}{W},(B,G) \\ \Rightarrow \\ \reg{\majtab{E}{t_{id}}{P'}}{T},(B,\maj{G.R}{req})}
\vspace{0.5cm}

EndEvent :
\inferrule
    { Head(B) = (tid, pid) \\ E[t_{id}] = \varepsilon \\ G.S = "event" }
    {\reg{E}{T}{W},({\concat{m}{B'}},G) \\ \Rightarrow \\ \reg{E}{T}{W},({B'},\maj{G.S}{"update"})}
\vspace{0.5cm}

\subsection{Actions sur les feux de signalisation}

ReqUpdate :
\inferrule
    { G.S = "update" }
    {(B,G.R = \concat{req}{Rq'}),S \\ \Rightarrow \\ (B,G.R = Rq'),S}
\vspace{0.5cm}

Update :
\inferrule
    { G.S = "update" \\ G.R = \varepsilon \\ S' = UpdateAllS(S,R.W,R.E) }
    {R,(B,G),S \\ \Rightarrow \\ R,(B,G.S = "none"),S'}
\vspace{0.5cm}

\end{document}