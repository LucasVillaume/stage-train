\documentclass[12pt]{article}
\usepackage{package/formal-grammar}
\usepackage{package/mathpartir}
\usepackage{package/algorithm}
\usepackage{package/algpseudocode}
\usepackage{amsmath}
\usepackage{mathtools}
\usepackage{graphicx}
\usepackage[dvipsnames]{xcolor}



\title{Composition de deux listes d'ordres}
\author{}

\begin{document}

\maketitle

\textbf{Hypothèse} : Les trains sont identiques dans les deux listes d'ordres.
Aucune disparition, aucune apparition de train.

\section{Propriétés nécéssaires à la composition}

\textbf{Symétrie des positions} : 
Soient $o1$ et $o2$ deux listes d'ordres.
Soit $F$ la liste des positions et directions finales de la liste d'ordre $o1$.
Soit $I$ la liste des positions et directions initiales de la liste d'ordre $o2$.
Si $F$ et $I$ sont égales, alors la propriété est vérifiée.
\vspace{0.5cm}\\
\textbf{Symétrie des aiguillages} : 
Soient $o1$ et $o2$ deux listes d'ordres.
Soit $S_1$ l'état des aiguillages à la fin de $o1$. 
Soit $S_2$ l'état des aiguillages au début de $o2$.
Si $S_1$ et $S_2$ sont égaux, alors la propriété est vérifiée.

\section{Cas complet vs Cas incomplet}
\subsection{Définition}
On dénombre deux cas ; le \textbf{cas complet} telque o1 et o2 n'ont pas de ressources critiques en commun,
et le \textbf{cas incomplet} telque o1 et o2 ont des ressources critiques en commun.
\vspace{0.5cm}\\
Par exemple, sur le circuit de la maquette :
\begin{itemize}
    \item o1 : $T_1 = 4* \rightarrow 8*$ et $T_2 = 5* \rightarrow 3*$ \\
          o2 : $T_1 = 8* \rightarrow 7*$ et $T_2 = 3* \rightarrow 3*$ \\
          Cas complet, pas de ressources critiques entre les listes, uniquement à l'intérieur de chacune d'elles.
    \item o1 : $T_1 = 4* \rightarrow 8*$ et $T_2 = 5* \rightarrow 7*$ \\
          o2 : $T_1 = 8* \rightarrow 7*$ et $T_2 = 7* \rightarrow 3*$ \\
          Cas incomplet, ressources critiques (canton 3) entre les deux listes d'ordres.
\end{itemize}
On fait face à un problème, comment distinguer les deux cas ? Pour l'exemple du cas simple,
chaque liste d'ordres possèdent localement le canton 3 comme ressources critiques. 
C'est-à-dire qu'elles gèrent cette ressource pour que ce ne soit pas un problème,
contrairement au cas incomplet.\\
Il faut lister les ressources critiques pour chaque couple de trains, puis examiner
si cette ressource pour ce couple est traitée localement dans o1.

\newpage

\subsection{Traitement des cas incomplet}
Une fois le cas incomplet détecté et tous les couples (trains/ressources critiques) identifiés,
il est possible de gérer la ressources critiques.
Il suffit d'ajouter des ordres incr et att dans les derniers events des trains concernés.
Si dans le premier events d'un train de o2 le pattern "turn(...);auth" est présent, il faut
alors ajouter un "turn" avant le "incr" précédent, en retirant ce pattern.
\\
\textbf{Exemple} :
\vspace{0.5cm}
\\
\fbox{%
\begin{minipage}{0.75\textwidth}
      o1 : $T_1 = 4* \rightarrow 8*$ / $T_2 = 5* \rightarrow 7*$\\
      T$_1$ : \{SU(L,8)\} / T$_2$ : \{SU(R,7)\}\\
      E : \{$\varepsilon$\}
      \vspace{0.4cm}\\
      o2 : $T_1 = 8* \rightarrow 7*$ / $T_2 = 7* \rightarrow 3*$\\
      T$_1$ : \{SU(R,3);SU(L,7)\} / T$_2$ : \{SU(R,4);SU(L,3)\}\\
      E : \{\{ev(T$_1$,1): att(3,1)\};
            \{ev(T$_2$,1): {\color{Red}turn(3,d);auth}\};...\}
\end{minipage}
}
\vspace{0.5cm}\\
Local$_{o1}$ : $\varepsilon$ / Local$_{o2}$ : (T1/T2) : 3 / Global : {\color{Cerulean}(T1$_{o1}$/T2$_{o2}$) : 3} \\
\textbf{Transformation}
\vspace{0.2cm}
\\
\fbox{%
\begin{minipage}{0.75\textwidth}
      o1 : $T_1 = 4* \rightarrow 8*$ / $T_2 = 5* \rightarrow 7*$\\
      T$_1$ : \{SU(L,8)\} / T$_2$ : \{SU(R,7)\}\\
      E : \{\{ev(T$_1$,3): {\color{Red}turn(3,d)};{\color{Cerulean}incr(3)}\};\{ev(T$_2$,2): {\color{Cerulean}att(3,1)}\}\}
      \vspace{0.4cm}\\
      o2 : $T_1 = 8* \rightarrow 7*$ / $T_2 = 7* \rightarrow 3*$\\
      T$_1$ : \{SU(R,3);SU(L,7)\} / T$_2$ : \{SU(R,4);SU(L,3)\}\\
      E : \{\{ev(T$_1$,1): att(3,1)\};...\}
\end{minipage}
}
\vspace{0.5cm}\\
On peut désormais appliquer les règles de transformation sur les deux listes d'ordres.

\section{Règles de transformation}
\subsection{Concernant les trains}

\textbf{Programme} :
Une fois la concaténation des deux programmes effectuée au sein d'un même train, si deux
"StartUntil" partagent la même direction, alors il faut {\color{Plum}écraser le premier au profit du second}. 

\subsection{Concernant le régulateur}
\textbf{Fusion des events} :
Les events de o1 peuvent être {\color{Plum}intégrés sans changement}. 
Soit $E_{x1}$ le nombre d'events pour le train x dans o1, alors les positions relatives
des events pour le train x de o2 {\color{Plum}augmentent de $E_{x1}$-1}.
Dans le cas où un event issue de o1 et un event issue de o2 doivent fusionner, il faut
faire attention à ce que les "turn" soient en premières positions.
\vspace{0.5cm}\\
\textbf{Jetons et attendre} :
{\color{Plum}Aucune modification} pour o1.
Pour o2, il faut changer les jetons pour "att" {\color{Plum}en se basant sur les valeurs courantes
à la fin de o1}. Ainsi, si $J[j] = x$ dans o1, alors \textit{"att(j,1)"} devient
\textit{"att(j,x+1)"}.



\end{document}

