
% Environments
\theoremstyle{plain}
\newtheorem{theorem}{Theorem}
\newtheorem{lemma}{Lemma}
\newtheorem{conjecture}{Conjecture}
\theoremstyle{definition}
\newtheorem{definition}{Définition}
\newtheorem{hypothesis}{Hypothèse}
\newtheorem{example}{Exemple}
\theoremstyle{remark}
\newtheorem*{inducthyp}{Induction Hypothesis}
\newtheorem*{notation}{Notation}
\newtheorem*{remarkThm}{Remark}
\newtheorem*{remarkThmRust}{\includegraphics[height=10pt]{Images/rust-logo-blk.png}~Remark}
\newtheorem{corollary}{Corollaire}
\newtheorem{ilemma}{Imported Lemma}

\newenvironment{IH}[1][]
{
	\begin{inducthyp}[#1]
		}{
			\hspace{\stretch{1}}\(\triangleright\)
	\end{inducthyp}
	}
\newenvironment{remarkRs}[1][]
{
	\begin{remarkThmRust}[#1]
		}{
			\hspace{\stretch{1}}\(\triangleleft\)
	\end{remarkThmRust}
	}
\newenvironment{remark}[1][]
{
	\begin{remarkThm}[#1]
		}{
			\hspace{\stretch{1}}\(\triangleleft\)
	\end{remarkThm}
	}

\tcbsetforeverylayer{
	enhanced,
	breakable,
	boxrule=0pt,
	frame hidden,
	sharp corners,
	left=1em,
	right=1em,
	}

\tcolorboxenvironment{definition}{
	borderline west={1pt}{0pt}{gray!80!blue},
	colback=gray!80!blue!5
}
\tcolorboxenvironment{theorem}{
	borderline west={1pt}{0pt}{gray!80!red},
	colback=gray!80!red!5
}
\tcolorboxenvironment{corollary}{
	borderline west={1pt}{0pt}{gray!80!red},
	colback=gray!80!red!5
}
\tcolorboxenvironment{lemma}{
	borderline west={1pt}{0pt}{gray!80!red},
	colback=gray!80!red!5
}
\tcolorboxenvironment{ilemma}{
	borderline west={1pt}{0pt}{gray!80!yellow},
	colback=gray!80!yellow!5
}
\tcolorboxenvironment{conjecture}{
	borderline west={1pt}{0pt}{gray!80!yellow},
	colback=gray!80!yellow!5
}
\tcolorboxenvironment{example}{
	borderline west={1pt}{0pt}{gray!80!green},
	colback=gray!80!green!10
}
\newtcolorbox{sidebox}{
	borderline={1pt}{0pt}{gray!80!cyan},
	colback=gray!80!cyan!15
	}
\newtcolorbox{warnbox}{
	borderline={1pt}{0pt}{gray!80!red},
	colback=gray!80!red!15
	}



\newenvironment{mathenum}
{
	\begin{enumerate}[label={{\rm(\roman*)}}, itemjoin={{; }}, itemjoin*={{; and }},]
}
	{\end{enumerate}}

\newenvironment{inlineenum}[1][and]
{
	\begin{enumerate*}[label={{\rm(\roman*)}}, before={}, itemjoin={{; }}, itemjoin*={{; #1 }},]
}
	{\end{enumerate*}}

\newenvironment{summary}[1]
{
	\vspace{\fill}
	\begin{center}
		\Large
		Summary -- #1
	\end{center}
}
{\afterpage{\restorepagecolor\pagestyle{myheadings}}}

\tcolorboxenvironment{summary}{
	borderline={1pt}{0pt}{gray!80!red},
	colback=gray!80!red!5,
}

\makeatletter
\newcommand{\sumoplus}{\DOTSB\sumoplus@\slimits@}
\newcommand{\sumoplus@}{\mathop{\mathpalette\sumoplus@@\relax}}
\newcommand{\sumoplus@@}[2]{% #1 is a style selection, #2 is unused
  \vcenter{\vbox{\halign{\hfil##\hfil\cr
    \scalebox{0.5}{$\m@th#1\boldsymbol{\oplus}$}\cr
    \noalign{\nointerlineskip}
    $\m@th#1\sum$\cr
  }}}%
}
\makeatother

% ------- Text
\newcommand{\ie}{i.e.\ }
\newcommand{\eg}{e.g.\ }
\newcommand{\etal}{et al.\ }
\newcommand{\etc}{etc.\ }
\newcommand{\name}[1]{#1}
\newcommand{\code}[1]{\texttt{#1}}

% ------- hyperlinks
\newcommand{\rulename}[1]{\hypertarget{#1}{(\textsc{#1})}}
\newcommand{\ruleref}[1]{\hyperlink{#1}{(\textsc{#1})}}
\newcommand{\defidx}[2]{\hypertarget{#1}{#1\index{#2}}}
\newcommand{\defref}[2]{\hyperlink{#2}{#1}}

% Notations
\newcommand{\llbracket}[0]{[\mkern-3mu[}
\newcommand{\rrbracket}[0]{]\mkern-3mu]}
\newcommand{\tuple}[1]{\langle #1\rangle}
\newcommand{\some}[1]{\lfloor #1\rfloor}
\newcommand{\ok}[1]{\lfloor #1\rfloor}
\newcommand{\err}[1]{\lceil #1\rceil}
\newcommand{\none}[0]{\bot}
\newcommand{\cfj}[0]{\textsc{cfj}}
\newcommand{\set}[1]{\mathbb{#1}}
\newcommand{\pfunction}[2]{#1 \nrightarrow #2}
\newcommand{\listtype}[1]{[#1]}
\newcommand{\multiset}[1]{\{\mkern-5mu\{#1\}\mkern-5mu\}}
\newcommand{\relation}[1]{\mathcal{#1}}
\newcommand{\powerset}[1]{\mathscr{P}(#1)}
\newcommand{\powermultiset}[1]{\mathscr{P}^\star(#1)}
\newcommand{\powerlist}[1]{\mathscr{P}^+(#1)}
\newcommand{\concat}[2]{#1 \concatop #2}
\newcommand{\append}[2]{#1 \appendop #2}
\newcommand{\multisetsum}[2]{#1 +^\star #2}
\newcommand{\existsunique}{\exists !}
\newcommand{\define}{\stackrel{\mathrm{\Delta}}{=}}
\newcommand{\slot}{[\cdot]}
\newcommand{\st}{\cdot}
\newcommand{\card}[1]{\left|#1 \right|}
\DeclareMathOperator{\appendop}{:\mkern-2mu :}
\DeclareMathOperator{\concatop}{@}
\DeclareMathOperator{\size}{size}
\DeclareMathOperator{\locations}{locations}
\DeclareMathOperator{\filter}{filter}
\DeclareMathOperator{\skl}{skl}
\DeclareMathOperator{\msg}{msg}
\DeclareMathOperator{\sortop}{sort}
\DeclareMathOperator{\applyop}{apply}
\DeclareMathOperator{\enqueueop}{enqueue}
\DeclareMathOperator{\pickop}{pick}
\DeclareMathOperator{\symdiff}{\ominus}

% LG 
\newcommand{\Compose}[2]{\prod_{#1}#2}
\newcommand{\compose}[2]{#1 \parallel #2}
\newcommand{\transSet}[1][u]{\mathscr{T}_{#1}}
\newcommand{\genTransSet}[1]{\mathscr{T}_{#1}}
\newcommand{\lgsize}[1]{\hyperlink{Location graph size}{\size}(#1)}
\newcommand{\lglocations}[1]{\hyperlink{Locations}{\locations}(#1)}
\newcommand{\lgextract}[1]{\hyperlink{Skeleton extraction}{e}(#1)}

\newcommand{\atomset}{\set{A}}
\newcommand{\prelocset}[0]{\hyperlink{Prelocation}{\set{L}^{\rm\sf p}}}
\newcommand{\lgraph}[3]{\mathtt{lgraph}(#1, #2, #3)}
\newcommand{\lauth}[2]{\mathtt{sAuth}(#1, #2)}
\newcommand{\llabel}[3]{\mathtt{sLabel}(#1, #2, #3)}
\newcommand{\ltrans}[2]{\mathtt{trans}(#1, #2)}

\newcommand{\seval}[1]{\hyperlink{Evaluation of matching interactions}{\mathtt{seval}(}#1\hyperlink{Evaluation of matching interactions}{)}}

\newcommand{\wellformed}[1]{\mathtt{WF}(#1)}
\newcommand{\auth}[0]{\hyperlink{Authorisation function}{\mathtt{Auth}}}
\newcommand{\lginstance}[3]{\tuple{#1, #2, #3}}
\newcommand{\lgreduce}[5][\transSet{}]{#2\vdash_{{#1}} #3 \xrightarrow{#4}#5}
\newcommand{\locreduce}[4]{#1\hyperlink{Unconstrained location transition}{\vartriangleright} #2 \xrightarrow{#3}#4}
\newcommand{\environment}[2]{#1\hyperlink{Environment}{\cdot} #2}
\newcommand{\envset}{\hyperlink{Environment}{\set{E}}}
\newcommand{\envunion}[2]{#1 \hyperlink{Environment union}{\cup} #2}
\newcommand{\envequiv}[2]{#1 \Bumpeq #2} %TODO
\newcommand{\genvequiv}[2]{#1 \Bumpeq #2} %TODO

% Generic elements
\newcommand{\sort}[1]{#1.\mathtt{sort}}
\newcommand{\proc}[1]{#1.\mathtt{proc}}
\newcommand{\prov}[1]{#1.\mathtt{provided}}
\newcommand{\req}[1]{#1.\mathtt{required}}
\newcommand{\bound}[1]{#1.\mathtt{bound}}
\newcommand{\unbound}[1]{#1.\mathtt{unbound}}
\newcommand{\pbound}[1]{#1.\mathtt{pbound}}
\newcommand{\rbound}[1]{#1.\mathtt{rbound}}
\newcommand{\punbound}[1]{#1.\mathtt{punbound}}
\newcommand{\runbound}[1]{#1.\mathtt{runbound}}
\newcommand{\supp}[1]{\hyperlink{Support}{\mathtt{supp}(}#1\hyperlink{Support}{)}}

%% Macros for locations
\newcommand{\lwellformed}[1]{\hyperlink{Well-formed prelocations}{\mathtt{WF}}(#1)}
\newcommand{\lproc}[1]{#1\hyperlink{Elements of a prelocation}{\mathtt{.proc}}}
\newcommand{\lsort}[1]{#1\hyperlink{Elements of a prelocation}{\mathtt{.sort}}}
\newcommand{\lprov}[1]{#1\hyperlink{Elements of a prelocation}{\mathtt{.provided}}}
\newcommand{\lreq}[1]{#1\hyperlink{Elements of a prelocation}{\mathtt{.required}}}
\newcommand{\locset}{\hyperlink{Location}{\set{L}}}
\newcommand{\location}[4]{\hyperlink{Location}{[}#1\hyperlink{Location}{:} #2 \hyperlink{Location}{\vartriangleleft} #3 \hyperlink{Location}{\bullet} #4\hyperlink{Location}{]}}

% Elements of location pregraphs
\newcommand{\prelgset}[0]{\hyperlink{Location pregraph}{\set{G}^{\rm\sf p}}}

% Elements of a graph
\newcommand{\gsort}[1]{#1\hyperlink{Elements of a location graph}{\mathtt{.sort}}}
\newcommand{\gprov}[1]{#1\hyperlink{Elements of a location graph}{\mathtt{.prov}}}
\newcommand{\greq}[1]{#1\hyperlink{Elements of a location graph}{\mathtt{.req}}}
\newcommand{\groles}[1]{#1\hyperlink{Elements of a location graph}{\mathtt{.roles}}}
\newcommand{\gbound}[1]{#1\hyperlink{Elements of a location graph}{\mathtt{.bound}}}
\newcommand{\gunbound}[1]{#1\hyperlink{Elements of a location graph}{.\mathtt{unbound}}}
\newcommand{\gpbound}[1]{#1\hyperlink{Elements of a location graph}{\mathtt{.pbound}}}
\newcommand{\gpunbound}[1]{#1\hyperlink{Elements of a location graph}{.\mathtt{punbound}}}
\newcommand{\grbound}[1]{#1\hyperlink{Elements of a location graph}{\mathtt{.rbound}}}
\newcommand{\grunbound}[1]{#1\hyperlink{Elements of a location graph}{.\mathtt{runbound}}}
\newcommand{\graphset}{\hyperlink{Location graph}{\set{G}}}
\newcommand{\gcompose}[2]{#1 \hyperlink{Location pregraph}{\parallel} #2}
\newcommand{\gequiv}[2]{#1 \hyperlink{Location pregraph structural equivalence}{\equiv} #2}

\newcommand{\separate}[2]{\hyperlink{Separated location pregraphs}{\mathtt{separate}}(#1, #2)}
\newcommand{\gwellformed}[1]{\hyperlink{Well-formed location pregraph}{\mathtt{WF}_G}(#1)}

% Macros for skeleton locations & location graphs
\newcommand{\sgraphset}{\hyperlink{Skeleton of a location graph}{\set{G}^{\rm\sf s}}}
\newcommand{\slocset}{\hyperlink{Skeleton of a location graph}{\set{L}^{\rm\sf s}}}
\newcommand{\skeleton}[1]{\hyperlink{Skeleton of a location graph}{\Sigma}(#1)}
\newcommand{\slocation}[3]{\hyperlink{Skeleton location graph}{[}#1 \hyperlink{Skeleton location graph}{\vartriangleleft} #2 \hyperlink{Skeleton location graph}{\bullet} #3\hyperlink{Skeleton location graph}{]}}
\newcommand{\sgcompose}[2]{#1 \hyperlink{Skeleton location graph}{\parallel} #2}
\newcommand{\sgequiv}[2]{#1 \hyperlink{Skeleton graph structural equivalence}{\equiv} #2}
% Element of a skeleton location graph
\newcommand{\sgsort}[1]{#1\hyperlink{Elements of a skeleton location graph}{\mathtt{.sort}}}
\newcommand{\sgprov}[1]{#1\hyperlink{Elements of a skeleton location graph}{\mathtt{.prov}}}
\newcommand{\sgreq}[1]{#1\hyperlink{Elements of a skeleton location graph}{\mathtt{.req}}}
\newcommand{\sgroles}[1]{#1\hyperlink{Elements of a skeleton location graph}{\mathtt{.roles}}}
\newcommand{\sgbound}[1]{#1\hyperlink{Elements of a skeleton location graph}{\mathtt{.bound}}}
\newcommand{\sgunbound}[1]{#1\hyperlink{Elements of a skeleton location graph}{.\mathtt{unbound}}}
\newcommand{\sgpbound}[1]{#1\hyperlink{Elements of a skeleton location graph}{\mathtt{.pbound}}}
\newcommand{\sgpunbound}[1]{#1\hyperlink{Elements of a skeleton location graph}{.\mathtt{punbound}}}
\newcommand{\sgrbound}[1]{#1\hyperlink{Elements of a skeleton location graph}{\mathtt{.rbound}}}
\newcommand{\sgrunbound}[1]{#1\hyperlink{Elements of a skeleton location graph}{.\mathtt{runbound}}}


% Elements of a (unconstrained) transition
\newcommand{\lbl}[1]{#1.\mathtt{label}}
\newcommand{\env}[1]{#1.\mathtt{env}}
\newcommand{\init}[1]{#1.\mathtt{init}}
\newcommand{\final}[1]{#1.\mathtt{final}}

% Macros for transitions
\newcommand{\indp}[6]{\hyperlink{Priority satisfaction of composed graphs (2)}{\mathtt{Ind}_P(}#1, #2, #3, #4, #5, #6\hyperlink{Priority satisfaction of composed graphs (2)}{)}}
\newcommand{\indi}[2]{\hyperlink{Independent interaction set}{\mathtt{Ind}_I(}#1, #2\hyperlink{Independent interaction set}{)}}
\newcommand{\condi}[4]{\hyperlink{Correct matching of a union of interaction sets}{\mathtt{Cond}_I(}#1, #2, #3, #4\hyperlink{Correct matching of a union of interaction sets}{)}}
\newcommand{\condp}[7]{\hyperlink{Priority satisfaction of composed graphs}{\mathtt{Cond}_P(}#1, #2, #3, #4, #5, #6, #7\hyperlink{Priority satisfaction of composed graphs}{)}}
\newcommand{\cond}[2]{\hyperlink{Correct environment}{\mathtt{Cond}(}#1, #2\hyperlink{Correct environment}{)}}

% Labels
\newcommand{\act}[1]{#1.\mathtt{act}} % À supprimer
\newcommand{\roles}[1]{#1.\mathtt{roles}}
\newcommand{\role}[1]{#1.\mathtt{role}}
\newcommand{\negate}[1]{\neg #1}
\newcommand{\conjugate}[1]{\overline{#1}}
\newcommand{\anydir}[1]{\hat{#1}}
\newcommand{\priority}[3]{#1: #2 \langle #3 \rangle}
\newcommand{\interaction}[3]{#1: #2 \langle #3 \rangle}
\newcommand{\priorityset}{\hyperlink{Priority constraint}{\set{\Pi}}}
\newcommand{\interactionset}{\hyperlink{Interaction}{\set{I}}}
\newcommand{\prequired}[1]{#1\hyperlink{Elements of priority constraints}{.\mathtt{required}}} % required roles of priority constraints
\newcommand{\pprovided}[1]{#1\hyperlink{Elements of priority constraints}{.\mathtt{provided}}} % provided roles of priority constraints
\newcommand{\proles}[1]{#1\hyperlink{Elements of priority constraints}{.\mathtt{roles}}} % roles of priority constraints
\newcommand{\irequired}[1]{#1\hyperlink{Elements of interactions}{.\mathtt{required}}} % required roles of interactions
\newcommand{\iprovided}[1]{#1\hyperlink{Elements of interactions}{.\mathtt{provided}}} % provided roles of interactions
\newcommand{\iroles}[1]{#1\hyperlink{Elements of interactions}{.\mathtt{roles}}} % roles of interactions
\newcommand{\prior}[1]{#1\hyperlink{Elements of a label}{.\mathtt{prior}}}
\newcommand{\sync}[1]{#1\hyperlink{Elements of a label}{.\mathtt{sync}}}
\newcommand{\laroles}[1]{#1\hyperlink{Elements of a label}{.\mathtt{roles}}} % roles of labels
\newcommand{\satisfies}[3]{#1 \hyperlink{Priority satisfaction}{\models}_{#2} #3}


\newcommand{\wf}[1]{\mathtt{WF}(#1)}

\newcommand{\names}[1]{#1\hyperlink{Elements of an environment}{.\mathtt{names}}} % Names of an environment
\newcommand{\graph}[1]{#1\hyperlink{Elements of an environment}{.\mathtt{graph}}} % Graph of an environment

\renewcommand{\owns}[0]{\multimap}

\newcommand{\observable}[2]{#1\downarrow_{#2}}

\newcommand{\rmv}[1]{\mathtt{rmv}(#1)}
\newcommand{\rmvself}[1]{\overline{\mathtt{rmv}}(#1)}
\newcommand{\rmvchan}{\mathtt{rmv}}
\newcommand{\rmvselfchan}{\overline{\mathtt{rmv}}}

\newcommand{\emptygraph}[0]{\emptyset}
\newcommand{\emptysgraph}[0]{\emptyset}
\newcommand{\subskeleton}[2]{#1 \hyperlink{Inclusion (Skeleton Location Graph)}{\subseteq} #2}
\newcommand{\sgraphunion}[2]{#1 \hyperlink{Union (Skeleton Location Graph)}{\cup} #2}

\newcommand{\gpartset}[1]{\hyperlink{Graph partition}{\set{P}}_{#1}} % The set of partitions of a given graph
\newcommand{\gpartfset}[1]{\hyperlink{Graph partitioning function}{\set{P}}_{#1}} % The set of partition functions (for an set of location graphs). 

\newcommand{\finer}[2]{#1\succ #2}
\newcommand{\coarser}[2]{#1\prec #2}

\newcommand{\pfiner}[2]{#1 \hyperlink{Finer graph partition}{\succ} #2}
\newcommand{\ffiner}[2]{#1 \hyperlink{Finer/Coarser partitioning functions}{\succ} #2}
\newcommand{\subgraph}[2]{#1 \hyperlink{Inclusion}{\subseteq} #2}
\newcommand{\gdistance}[2]{d(#1, #2)}
\newcommand{\pdistance}[2]{D(#1, #2)}

\newcommand{\makeaction}[3]{#1:#2\langle#3\rangle}
\newcommand{\makeprior}[3]{#1:#2\langle#3\rangle}

\newcommand{\depth}[0]{\mathtt{depth}}

% LG nesting
\newcommand{\nest}[1]{\hyperlink{Location nesting function}{n(}#1\hyperlink{Location nesting function}{)}}
\newcommand{\nestr}[1]{\hyperlink{Inverse of location nesting}{n^{-1}(}#1\hyperlink{Inverse of location nesting}{)}}
\newcommand{\Nest}[2]{\hyperlink{Graph nesting function}{N_{#1}(}#2\hyperlink{Graph nesting function}{)}}
\newcommand{\Nestr}[2]{\hyperlink{Inverse of location nesting}{N^{-1}_{#1}(}#2\hyperlink{Inverse of location nesting}{)}}
\newcommand{\nauth}{\hyperlink{}{\mathtt{Auth}_2}} % Authorisation function for 2nd order graphs.
\newcommand{\ninter}[4]{\hyperlink{Second order interactions}{\mathtt{inter}_2(}#1, #2, #3, #4\hyperlink{Second order interactions}{)}}

% LG extended simulation 

\newcommand{\simueq}[1]{\eqcirc_{#1}}
\newcommand{\simulates}[2]{#1\sim #2}

% LG merging
\newcommand{\lgmerge}[2]{#1\star #2}
\newcommand{\lifttop}[1]{#1^\top}
\newcommand{\liftbot}[1]{#1^\bot}
\newcommand{\lift}[1]{#1^{\top\bot}}
\newcommand{\wfmerge}[1]{\mathtt{WF}_\star(#1)}
\newcommand{\side}[1]{\mathtt{side}(#1)}
\newcommand{\proj}[2]{\pi_{{#1}}(#2)}

% Nesting 
\newcommand{\prune}[1]{\hyperlink{Label pruning}{\mathtt{prune}(}#1\hyperlink{Label pruning}{)}}
\newcommand{\recover}[2]{\hyperlink{Graph recovery}{\mathtt{recover}(}#1, #2\hyperlink{Graph recovery}{)}}
% Nesting (recursive -- naive)
\newcommand{\recnestloc}[2]{\hyperlink{Recursive location nesting function}{\mathbb{n}}_{#1}(#2)}
\newcommand{\recnest}[2]{\hyperlink{Recursive graph nesting function}{\mathbb{N}}_{#1}(#2)}
% recursive -- not naive
\newcommand{\rnest}[2]{\hyperlink{Higher-order nesting function}{\nu_{#1}(#2)}}

% Strict encapsulation policy
\newcommand{\sownership}[2]{#1\hyperlink{Strict ownership domain}{\#} #2}
\newcommand{\seownership}[2]{#1\hyperlink{Strict ownership domain}{\star } #2}
\newcommand{\sgsownership}[2]{#1\hyperlink{Ownership domains of skeleton graphs}{\#_s} #2}
\newcommand{\sgseownership}[2]{#1\hyperlink{Ownership domains of skeleton graphs}{\star_s } #2}
\newcommand{\sesort}{\hyperlink{sesort}{\set{S}_{se}}}
\newcommand{\sesortid}[1]{#1 \hyperlink{sesortid}{{\tt .id}}}
\newcommand{\sesortowned}[1]{#1 \hyperlink{sesortowned}{{\tt .owned}}}
\newcommand{\selowned}[1]{#1 \hyperlink{selowned}{{\tt .owned}}}
\newcommand{\segowned}[1]{#1 \hyperlink{selowned}{{\tt .owned}}}
\newcommand{\selid}[1]{#1 \hyperlink{selowned}{{\tt .id}}}
\newcommand{\seownerpred}[1]{#1 \hyperlink{seownerpred}{{\tt .is\_owner}}}
\newcommand{\sewf}[1]{\hyperlink{sewf}{{\tt WF}_{se}(}#1\hyperlink{sewf}{)}}
\newcommand{\selocs}{\hyperlink{sewf}{\set{L}_{se}}}
\newcommand{\segraphs}{\hyperlink{sewf}{\set{G}_{se}}}
\newcommand{\seowns}[2]{#1 \hyperlink{Strict ownership relation}{\owns} #2}
\newcommand{\srange}[2]{\hyperlink{Label range}{{\tt range}(}#1, #2\hyperlink{Label range}{)}}

% Fip policy
\newcommand{\newarrow}[1]{\mathtt{new\_arrow}(#1)}
\newcommand{\delarrow}[1]{\mathtt{del\_arrow}(#1)}

% -------------------------


% JavaLoc
\newcommand{\classname}[1]{\texttt{#1}}
\newcommand{\methodname}[1]{\texttt{#1}}


% CFJ single thread semantics
\newcommand{\cfjstsreduce}[1]{\overset{#1}{\longrightarrow}}

% CFJ concurrent semantics
\newcommand{\cfjreduce}[0]{\Longrightarrow}

% FIP commands
\newcommand{\coddep}[0]{\dashleftarrow}
\newcommand{\fip}[0]{{\textsc{fip}}}
\newcommand{\cod}[0]{{\textsc{cod}}}
\newcommand{\lock}[0]{\text{\faLock}}
\newcommand{\unlock}[0]{\text{\faUnlock}}


% -------------------------
% Abstract Machine
\newcommand{\amprov}{{\tt provided}}
\newcommand{\amreq}{{\tt required}}
\newcommand{\ambind}[3]{\hyperlink{Transition item}{{\tt Bind}(}#1, #2, #3\hyperlink{Transition item}{)}}
\newcommand{\amcreate}[1]{\hyperlink{Transition item}{{\tt Create}(}#1\hyperlink{Transition item}{)}}
\newcommand{\amexpect}[3]{\hyperlink{Transition item}{{\tt Expect}(}#1, #2, #3\hyperlink{Transition item}{)}}
\newcommand{\amreceive}[2]{\hyperlink{Transition item}{{\tt Receive}(}#1, #2\hyperlink{Transition item}{)}}
\newcommand{\amrelease}[3]{\hyperlink{Transition item}{{\tt Release}(}#1, #2, #3\hyperlink{Transition item}{)}}
\newcommand{\amsend}[3]{\hyperlink{Transition item}{{\tt Send}(}#1, #2, #3\hyperlink{Transition item}{)}}
\newcommand{\amsort}[2]{\hyperlink{Transition item}{{\tt Sort}(}#1, #2\hyperlink{Transition item}{)}}
\newcommand{\amremove}[1]{\hyperlink{Transition item}{{\tt Remove}(}#1\hyperlink{Transition item}{)}}
\newcommand{\amtransitem}{\hyperlink{Transition item}{\set{T}_i}}
\newcommand{\amtrans}{\hyperlink{Transition (Abstract machine)}{\set{T}}}
\newcommand{\amtransenqueue}[2]{\hyperlink{Transition item enqueuing}{\enqueueop(}#1, #2\hyperlink{Transition item enqueuing}{)}}
\newcommand{\amfilter}[1]{\hyperlink{Transition filter}{\filter(}#1\hyperlink{Transition filter}{)}}
\newcommand{\amskli}[1]{\hyperlink{Skeleton of a transition item}{\skl_i(}#1\hyperlink{Skeleton of a transition itemSkeleton of a transition item}{)}}
\newcommand{\amskl}[1]{\hyperlink{Skeleton of a transition}{\skl(}#1\hyperlink{Skeleton of a transition}{)}}
\newcommand{\amwftrans}[1]{\hyperlink{Well-formed transition (Abstract machine)}{{\tt WF}_t(}#1\hyperlink{Well-formed transition (Abstract machine)}{)}}
\newcommand{\amnotintransset}{{\tt Not\_In\_Trans\_Set}}
\newcommand{\amnotselected}{{\tt Not\_Selected}}
\newcommand{\amnotallowed}{{\tt Not\_allowed}}
\newcommand{\amnomatch}{{\tt No\_match}}
\newcommand{\amwfhandle}[1]{\hyperlink{Well-formed handle}{{\tt WF}_h(}#1\hyperlink{Well-formed handle}{)}}
\newcommand{\ammsg}[1]{\hyperlink{Messages of a transition}{\msg(}#1\hyperlink{Messages of a transition}{)}}
\newcommand{\amsortrel}[2]{\hyperlink{New sort in a transition}{\sortop(}#1, #2\hyperlink{New sort in a transition}{)}}
\DeclareMathOperator{\ep}{endpoints}
\newcommand{\amendpoints}[2]{\hyperlink{Endpoints effects of a transition}{\ep(}#1, #2\hyperlink{Endpoints effects of a transition}{)}}
\newcommand{\amretaddr}{{\tt ret}}
\newcommand{\amapplyloc}[3]{\hyperlink{Application of a transition to a location}{\applyop_l(}#1, #2, #3\hyperlink{Application of a transition to a location}{)}}
\newcommand{\aminitstack}{{\tt stack}_i}
\newcommand{\amrolesset}{\hyperlink{sec:am:base_sets}{\set{R}}}
\newcommand{\amdirset}{\hyperlink{sec:am:base_sets}{\set{D}}}
\newcommand{\ammsgset}{\hyperlink{sec:am:base_sets}{\set{M}}}
\newcommand{\amsortsset}{\hyperlink{sec:am:base_sets}{\set{S}}}
\newcommand{\amepset}{\hyperlink{Endpoint}{\set{E}}}
\newcommand{\amsklocset}{\hyperlink{Skeleton locations (Abstract machine)}{\set{\Sigma}_l}}
\newcommand{\amskgset}{\hyperlink{Skeleton location graphs (Abstract machine)}{\set{\Sigma}}}
\newcommand{\amsilentprimitive}[0]{{\sl silent}}
\newcommand{\ambindprimitive}[2]{{\sl bind}(#1, #2)}
\newcommand{\amreleaseprimitive}[1]{{\sl release}(#1)}
\newcommand{\amcreateprimitive}[1]{{\sl create}(#1)}
\newcommand{\amnewsortprimitive}[1]{{\sl newSort}(#1)}
\newcommand{\amreceiveprimitive}[1]{{\sl receive}(#1)}
\newcommand{\amsendprimitive}[2]{{\sl send}(#1, #2)}
\newcommand{\amexpectprimitive}[2]{{\sl expect}(#1, #2)}
\newcommand{\amendprimitive}{{\sl end}}
\newcommand{\amresetprimitive}{{\sl reset}}
\newcommand{\amnexttransprimitive}{{\sl next\_transition}}
\newcommand{\amcommitprimitive}{{\sl commit}}
\newcommand{\amprimitiveset}{\hyperlink{Primitives of the abstract machine}{\set{P}}}
\DeclareMathOperator{\statenext}{next}
\newcommand{\amnext}[1]{\statenext(#1)}
\DeclareMathOperator{\statememory}{memory}
\newcommand{\ammemory}[1]{\statememory(#1)}
\newcommand{\ammemoryupdate}[3]{#1[#2 \mapsto #3]}
\newcommand{\ammemoryremove}[2]{#1[\times #2]}
\newcommand{\amstatesset}{\set{S}_t}
\newcommand{\aminitialstate}{\sigma_i}
\newcommand{\amobjects}{\hyperlink{Objects of the abstract machine}{\set{O}}}
\newcommand{\amaddressesset}{\set{A}}
\newcommand{\amlochandlesset}{\hyperlink{Location handle}{\set{L}_H}}
\newcommand{\amlocset}{\hyperlink{Location (Abstract machine)}{\set{L}}}
\newcommand{\amtransitemsum}[2]{#1 \hyperlink{Transition item sum}{\oplus} #2}
\newcommand{\amtranssum}[2]{#1 \hyperlink{Transition sum}{\oplus} #2}
\newcommand{\amtransSum}[2]{\sumoplus_{#1} #2}
\newcommand{\amerrorset}{\hyperlink{Transition result}{\set{E}_r}}
\newcommand{\amsuccessset}{\hyperlink{Transition result}{\set{S}_r}}
\newcommand{\amresultset}{\hyperlink{Transition result}{\set{R}_t}}
\newcommand{\amts}[2]{ts(#1, #2)}
\newcommand{\amaf}[3]{af(#1, #2, #3)}
\DeclareMathOperator{\complete}{complete}
\newcommand{\amcomplete}[1]{\hyperlink{Complete set of transition}{\complete(}#1\hyperlink{Complete set of transition}{)}}
\newcommand{\ampick}[2]{\hyperlink{Picking relation}{\pickop(}#1, #2\hyperlink{Picking relation}{)}}
\DeclareMathOperator{\exchange}{exchange}
\newcommand{\amexchange}[1]{\hyperlink{Exchange function}{\exchange(}#1\hyperlink{Exchange function}{)}}
\newcommand{\amexchanget}[2]{\hyperlink{Exchange function}{\exchange_t(}#1, #2\hyperlink{Exchange function}{)}}
\DeclareMathOperator{\items}{items}
\newcommand{\amitems}[1]{\hyperlink{Role predicate}{\items(}#1\hyperlink{Role predicate}{)}}
\DeclareMathOperator{\roleok}{roles\_ok}
\newcommand{\amroleok}[2]{\hyperlink{Role predicate}{\roleok(}#1, #2\hyperlink{Role predicate}{)}}
\newcommand{\amapplyt}[2]{\hyperlink{Transition application to a skeleton location}{\applyop_t(}#1, #2\hyperlink{Transition application to a skeleton location}{)}}
\newcommand{\amapply}[2]{\hyperlink{Transition application to a skeleton location graph}{\applyop(}#1, #2\hyperlink{Transition application}{)}}
\newcommand{\amapplyf}[3]{\hyperlink{Full transition application to a skeleton location graph}{\applyop_f(}#1, #2, #3\hyperlink{Full transition application}{)}}
\newcommand{\amwflg}[1]{\hyperlink{Well-formed location graph}{{\tt WF}_g(}#1\hyperlink{Well-formed location graph}{)}}
\DeclareMathOperator{\result}{result}
\DeclareMathOperator{\possibleresult}{possible\_result}
\newcommand{\amlistres}[7]{\hyperlink{Results of a list of transitions}{\result(}#1, #2, #3, #4, #5, #6, #7\hyperlink{Results of a list of transitions}{)}}
\newcommand{\amlistpres}[3]{\hyperlink{Possible results of a list of transitions}{\possibleresult(}#1, #2, #3\hyperlink{Possible results of a list of transitions}{)}}

% Ad-hoc policy in LG
\newcommand{\lsindex}{ls}
\newcommand{\lswf}[1]{{\tt WF}(#1)}
\newcommand{\lsgraphset}{\mathbb{G}_{ls}}
\renewcommand{\S}{{\tt S}}
\renewcommand{\L}{{\tt L}}
\newcommand{\lssortset}{\set{S}_{\lsindex}}
\newcommand{\lscreate}[1]{{\tt create}(#1)}
\newcommand{\lsremove}[1]{{\tt remove}(#1)}
\newcommand{\lsroleset}{\set{R}_{\lsindex}}
\newcommand{\lstransset}{\transSet[ls]}


% PubSub Server
\newcommand{\tcploc}{{\tt TCPLoc}}
\newcommand{\tcpread}{{\tt TCPRead}}
\newcommand{\tcplistener}{{\tt Listener}}
\newcommand{\subtotcp}{{\tt S2T}}
\newcommand{\tcptosub}{{\tt T2S}}


% from : https://tex.stackexchange.com/questions/365952/enumerate-a-list-of-equations#365956
\makeatletter
\newcommand{\leqnomode}{\tagsleft@true}
\makeatother
\newcounter{dummy}

\renewcommand{\theequation}{\arabic{equation}}
\newenvironment{enumeq}{
	\bgroup
	\leqnomode
	\setcounter{dummy}{\theequation}
	\setcounter{equation}{0}
	\renewcommand{\theequation}{\roman{equation}}
}{
	\setcounter{equation}{\thedummy}
	\egroup
}
