\documentclass{article}


\begin{document}

\section{Introduction}
    \subsection{Problématique et objectifs}
    \subsection{Outils}
    \subsection{Preview des résultats}
    À la fin, le lecteur connaît l'objectif principal, le principe de la modélisation, TLA+ (mention rapide des 
    automates et des LTS). On fait aussi mention des résultats obtenus (modèle, implem, expériences, ...)

\section{Modélisation}
    Faire un petit avant-propos pour parler du python : d'abord construit un petit programme python pour vérifier que
    TLA+ faisait bien ce qu'on voulait.
    \subsection{Premières approches}
        \subsubsection{Modèle 0}
        \subsubsection{Modèle 1}
        \subsubsection{Modèle 2}
        On présente rapidement les premiers modèles, notamment les choix de modélisation et les changements majeurs. 
        Fait des petits liens avec le modèle final. (Buffer dans le modele final pcq besoin de séparer reg et gamma)
    \subsection{Modèle "final"}
    \subsection{Comparaison}
    On passe rapidement sur les premiers modele en faisant le lien avec le modèle final. Puis on présente en profondeur
    le modèle final, en insistant sur les choix de modélisation.
    On aborde également les règles de transition et les propriétés (+ une rapide
    comparaison des modèles : taille du graphe, temps de calcul).

\section{Implémentation}
    \subsection{En TLA+}
    [Pas certain] Partie assez courte sur la manière dont on gère l'implémentation en TLA (Spec, Properties, etc...).
    \subsection{Fonction de composition de chemins}
    \subsection{Expériences sur Grid5000}
    À la fin, le lecteur connaît le rôle de la fonction de composition, la notion d'état élémentaire 
    et les résultats des expériences. 

\section{Quêtes secondaires}
    \subsection{Maquette}

\section{Conclusion et perspectives}

\end{document}